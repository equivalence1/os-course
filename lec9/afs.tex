\begin{frame}
\frametitle{Andrew File System}
\begin{itemize}
  \item<1-> Andrew File System - ФС разработанная в CMU в 1980-х:
    \begin{itemize}
      \item основная цель - масштабирование;
    \end{itemize}
  \item<2-> Принцип работы очень простой:
    \begin{itemize}
      \item файлы \emph{целиком} кешируются локально;
      \item если файл не в кеше запрашиваем файл с сервера - передаем полное имя файла;
      \item если файл в кеше проверяем валидность копии и запрашиваем новую если нужно;
      \item когда закончили обновляем версию на сервере;
    \end{itemize}
\end{itemize}
\end{frame}

\begin{frame}
\frametitle{Проблемы AFS}
\begin{itemize}
  \item<1-> Сервер должен для каждого клиента выполнять поиск по имени:
    \begin{itemize}
      \item поиск может приводить к нескольким обращениям к диску;
      \item если клиентов много поиск может тратить много ресурсов CPU.
    \end{itemize}
  \item<2-> Серевер может быть загружен запросами на проверку версии файла:
    \begin{itemize}
      \item клиенты должны проверять валидность своей копии файла;
      \item даже если файл изменяется редко, проверять его валидность нужно всегда;
    \end{itemize}
\end{itemize}
\end{frame}

\begin{frame}
\frametitle{Решения проблем AFS}
\begin{itemize}
  \item<1-> Callback - сервер сам нотифицирует об изменениях файла:
    \begin{itemize}
      \item клиенты не заваливают сервер запросами на проверку валидности копии;
      \item сервер рассылает сообщения заинтересованным клиентам, когда копия обновилась;
    \end{itemize}
  \item<2-> Поиск файла по имени перекладывается на клиента:
    \begin{itemize}
      \item клиент делает запрос к серверу на каждый каталог в имени;
      \item клиент кеширует ответы от сервера и тем самым уменьшает нагрузку;
      \item на каждый полученный каталог регистрируется Callback;
    \end{itemize}
\end{itemize}
\end{frame}

\begin{frame}
\frametitle{Финальные замечания про AFS}
\begin{itemize}
  \item AFS определяет протокол взаимодействия клиента и сервера:
    \begin{itemize}
      \item NFS (стандарт де-факто) тоже определяет протокол;
      \item протокол не приближает нас к реализации серверной части;
      \item как распределять информацию по среверам?
      \item как бороться с отказами на сервере?
    \end{itemize}
\end{itemize}
\end{frame}
