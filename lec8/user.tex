\begin{frame}
\frametitle{Файловая система}
\begin{itemize}
  \item<1-> Файловая система - иерархическая организация чего-либо в файлы и каталоги:
    \begin{itemize}
      \item файл - последовательность байт;
      \item каталог - набор файлов и других каталогов;
    \end{itemize}
  \item<2-> Файл - удобная абстракция:
    \begin{itemize}
      \item чтение/запись последовательности байт очень универсальный интерфейс:
        \begin{itemize}
          \item работа с сокетом - чтение/запись последовательности байт;
          \item устройства могут быть представлены как файлы (/dev/ttyACM0, /dev/input/mouse);
          \item порции данных на диске могут быть организованы в файлы;
          \item UNIX - все есть файл.
        \end{itemize}
    \end{itemize}
\end{itemize}
\end{frame}

\begin{frame}
\frametitle{Типичный интерфейс файловой системы}
\begin{itemize}
  \item open/create - открытие и создание файла;
  \item read/write/seek - чтение/запись/позиционирование файла;
  \item close - закрытие файла;
  \item link/unlink - создание и удаление жесткой ссылки на файл;
  \item mkdir - создание каталога;
  \item opendir - открытие каталога;
  \item readdir - чтение содержимого каталога;
  \item closedir - закрытие каталога;
\end{itemize}
\end{frame}

\begin{frame}
\frametitle{Расширенный интерфейс файловой системы}
\begin{itemize}
  \item Зачастую современные ФС предоставляют много других возможностей:
    \begin{itemize}
      \item truncate - задание размера файла и punch hole;
      \item дефрагментация - переупорядочивание содержимого ФС на диске;
      \item шифрование;
      \item списки контроля доступа;
      \item RAID;
      \item и многое другое...
    \end{itemize}
\end{itemize}
\end{frame}

\begin{frame}
\frametitle{Файловый дескриптор}
\begin{itemize}
  \item<1-> В UNIX есть два понятия: File Descript\emph{or} и File Descript\emph{ion}:
    \begin{itemize}
      \item File Descript\emph{or} с точки зрения программиста - просто число идентифицирующее открытый файл;
      \item можете смотреть на него как на индекс в таблице открытых файлов процесса;
      \item дескриптор имеет смысл только в рамках процесса;
    \end{itemize}
  \item<2-> File Descript\emph{or} - это ссылка на File Descript\emph{ion}:
    \begin{itemize}
      \item дескриптор у каждого процесса свой, но они могут ссылаться на один и тот же открытый файл;
      \item File Descript\emph{ion} в частности хранить текущее смещение в файле - могут быть неприятности после fork;
      \item чтобы создать новый File Descript\emph{ion} нужно заново открыть файл.
    \end{itemize}
\end{itemize}
\end{frame}

\begin{frame}
\frametitle{Задача файловой системы}
\begin{itemize}
  \item<1-> Основная задача ФС сопоставить имени файла и смещению в файле смещение на диске
    \begin{itemize}
      \item ФС должна поддерживать информацию о свободном/занятом месте на диске и выделять/освобождать место по требованию;
      \item ничего не напоминает?
    \end{itemize}
  \item<2-> ФС - это еще один алокатор, но со своей спецификой:
    \begin{itemize}
      \item все запросы к диску тоже проходят через ФС;
      \item мы не можем полагаться на надежность процессов;
      \item выделенное место не непрерывно на диске;
      \item от ФС часто ожидают надежного хранения;
    \end{itemize}
\end{itemize}
\end{frame}
